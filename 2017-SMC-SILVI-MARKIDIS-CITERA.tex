% -------------------------------------------------------------------------------------------
% Template for SMC 2016 ---------------------------------------------------------------------
% adapted from the template for SMC 2012 and 2011, which were adapted from that of SMC 2010 -
% -------------------------------------------------------------------------------------------
% -------------------------------------------------------------------------------------------

\documentclass{article}
\usepackage{smc2017}
\usepackage{times}
\usepackage{ifpdf}
\usepackage[english]{babel}
\usepackage{cite}

%%%%%%%%%%%%%%%%%%%%%%%% Some useful packages %%%%%%%%%%%%%%%%%%%%%%%%%%%%%%%
%%%%%%%%%%%%%%%%%%%%%%%% See related documentation %%%%%%%%%%%%%%%%%%%%%%%%%%
%\usepackage{amsmath} % popular packages from Am. Math. Soc. Please use the 
%\usepackage{amssymb} % related math environments (split, subequation, cases,
%\usepackage{amsfonts}% multline, etc.)
%\usepackage{bm}      % Bold Math package, defines the command \bf{}
\usepackage{paralist} % extended list environments
%%subfig.sty is the modern replacement for subfigure.sty. However, subfig.sty 
%%requires and automatically loads caption.sty which overrides class handling 
%%of captions. To prevent this problem, preload caption.sty with caption=false 
%\usepackage[caption=false]{caption}
%\usepackage[font=footnotesize]{subfig}

%user defined variables
\def\papertitle{OMNIDIRECTIONAL POINTS OF LISTENING}
\def\firstauthor{Giuseppe Silvi}
\def\secondauthor{Marco Matteo Markidis}
\def\thirdauthor{Pasquale Citera}

% adds the automatic
% Saves a lot of ouptut space in PDF... after conversion with the distiller
% Delete if you cannot get PS fonts working on your system.

% pdf-tex settings: detect automatically if run by latex or pdflatex
\newif\ifpdf
\ifx\pdfoutput\relax
\else
   \ifcase\pdfoutput
      \pdffalse
   \else
      \pdftrue
\fi

\ifpdf % compiling with pdflatex
  \usepackage[pdftex,
    pdftitle={\papertitle},
    pdfauthor={\firstauthor, \secondauthor, \thirdauthor},
    bookmarksnumbered, % use section numbers with bookmarks
    pdfstartview=XYZ % start with zoom=100% instead of full screen; 
                     % especially useful if working with a big screen :-)
   ]{hyperref}
  %\pdfcompresslevel=9

  \usepackage[pdftex]{graphicx}
  % declare the path(s) where your graphic files are and their extensions so 
  %you won't have to specify these with every instance of \includegraphics
  \graphicspath{{./figures/}}
  \DeclareGraphicsExtensions{.pdf,.jpeg,.png}

  \usepackage[figure,table]{hypcap}

\else % compiling with latex
  \usepackage[dvips,
    bookmarksnumbered, % use section numbers with bookmarks
    pdfstartview=XYZ % start with zoom=100% instead of full screen
  ]{hyperref}  % hyperrefs are active in the pdf file after conversion

  \usepackage[dvips]{epsfig,graphicx}
  % declare the path(s) where your graphic files are and their extensions so 
  %you won't have to specify these with every instance of \includegraphics
  \graphicspath{{./figures/}}
  \DeclareGraphicsExtensions{.eps}

  \usepackage[figure,table]{hypcap}
\fi

%setup the hyperref package - make the links black without a surrounding frame
\hypersetup{
    colorlinks,%
    citecolor=black,%
    filecolor=black,%
    linkcolor=black,%
    urlcolor=black
}

\hyphenation{so-ur-ces}
\hyphenation{el-ec-tro-aco-us-ti-cal}


% Title.
% ------
\title{\papertitle}

% Authors
% Please note that submissions are NOT anonymous, therefore 
% authors' names have to be VISIBLE in your manuscript. 
%
% Single address
% To use with only one author or several with the same address
% ---------------
%\oneauthor
%   {\firstauthor} {Affiliation1 \\ %
%     {\tt \href{mailto:author1@smcnetwork.org}{author1@smcnetwork.org}}}

%Two addresses
%--------------
% \twoauthors
%   {\firstauthor} {Affiliation1 \\ %
%     {\tt \href{mailto:author1@smcnetwork.org}{author1@smcnetwork.org}}}
%   {\secondauthor} {Affiliation2 \\ %
%     {\tt \href{mailto:author2@smcnetwork.org}{author2@smcnetwork.org}}}

% Three addresses
% --------------
 \threeauthors
   {\firstauthor} {Conservatorio S. Cecilia di Roma \\ %
     {\tt \href{mailto:me@giuseppesilvi.com}{me@giuseppesilvi.com}}}
   {\secondauthor} {nonoLab \\ %
     {\tt \href{mailto:mm.markidis@gmail.com}{mm.markidis@gmail.com}}}
   {\thirdauthor} { EarShot srls \\ %
     {\tt \href{mailto:pasqualecitera81@gmail.com}{pasqualecitera81@gmail.com}}}

% -------------------------------------------------------------------------------------------
% ************************************************** the document starts here ***************
% -------------------------------------------------------------------------------------------
\begin{document}
%

%++++++++++++++++++++++++++++++++++++++++++++++++++++++++++++++++++++++++++++++++++++++++++++
%+++++++++++++++++++++++++++++++++++++++++++++++++++++++++++++++++++++++++++ TO DO LIST ++++
%++++++++++++++++++++++++++++++++++++++++++++++++++++++++++++++++++++++++++++++++++++++++++
\textbf{to do list}

\begin{compactitem}
	\item A to A module
	\item comparison omni, stereo, soundfield?
	\item octave plot of shape with pitch colours
	\item path to loudness
	\item shape of chromatic scale
	\item shape of density
	\item shape of berio seq IXb
	\item mono to stereo strategy
\end{compactitem}
\clearpage
%
\capstartfalse
\maketitle
\capstarttrue

%++++++++++++++++++++++++++++++++++++++++++++++++++++++++++++++++++++++++++++++++++++++++++++
%+++++++++++++++++++++++++++++++++++++++++++++++++++++++++++++++++++++++++++++ ABSTRACT ++++
%++++++++++++++++++++++++++++++++++++++++++++++++++++++++++++++++++++++++++++++++++++++++++

\begin{abstract}

Place your abstract at the top left column on the first page.
Please write about 150--200 words that specifically highlight the purpose of your work,
its context, and provide a brief synopsis of your results.
Avoid equations in this part.

\end{abstract}

%++++++++++++++++++++++++++++++++++++++++++++++++++++++++++++++++++++++++++++++++++++++++++++
%+++++++++++++++++++++++++++++++++++++++++++++++++++++++++++++++++++++++++ INTRODUCTION ++++
%++++++++++++++++++++++++++++++++++++++++++++++++++++++++++++++++++++++++++++++++++++++++++

\section{Introduction}
\label{sec:intro}

The core of this research is about the \emph{equilibrium} of acoustical and electroacoustical
sources for music performances and mixed media concert environment.

The starting point is the assumption that there are not shared characteristics from both
acoustical and electroacoustical sources. Acoustical sources, their pressure and power on air,
came to ears through the full space in a \emph{transparent} way. Loudspeaker, the best one,
push air in partial space in a \emph{visible} way.

Acoustic differences from mixed sources may cause distraction and distortion of acoustic reality. 

First remarkable step into this conception is clarify the stereo concept by the assumption that
we consider stereo a listening attitude an call stereo a technology able to reproduce sound with
directional informations and the illusion of audible perspective.

Stereo is a listening attitude and not a reproduction channel array.

%++++++++++++++++++++++++++++++++++++++++++++++++++++++++++++++++++++++++++++++++++++++++++++
%++++++++++++++++++++++++++++++++++++++++++++++++++++++++++++++++++++++++++++++++ ROOTS ++++
%++++++++++++++++++++++++++++++++++++++++++++++++++++++++++++++++++++++++++++++++++++++++++

\section{Historical Roots}
\label{sec:roots}

In the beginning was the tetrahedron. Despite its greek roots, we know the Tetrahedron
only through Michael Gerzon \cite{mgamb02} postulations. Michael Gerzon, better than
Pitagora and Euclide, understood the efficiencies of tetrahedral shape for omnidirectional
sound description. 

\textbf{small description of what of ambisonic will referenced }

The \emph{S.T.ONE} project (\emph{Spherical Tetrahedral ONE}) grew up from the need of an
electroacoustic environment in which acoustical and electronic instruments have the same
impact on listeners.

A traditional loudspeaker could be controlled only on dynamic dimension of power.
This power control can't establish complete balance of each aspect of sound propagation for
acoustical instruments. Each acoustical instrument have a complex propagation that
combine dynamics, timbre and sound shape relationship. This difference, between instruments
and traditional loudspeaker is responsible of unglue from acoustics and electronics.

\emph{S.T.ONE} is a tetrahedral loudspeaker able to reproduce spherical sound field. to be conitinued?

\emph{TETRAREC} is a spaced non-directional microphone technique based on tetrahedral shape. to be continued? 

%++++++++++++++++++++++++++++++++++++++++++++++++++++++++++++++++++++++++++++++++++++++++++++
%++++++++++++++++++++++++++++++++++++++++++++++++++++++++++++++++++++++++++++++++ STONE ++++
%++++++++++++++++++++++++++++++++++++++++++++++++++++++++++++++++++++++++++++++++++++++++++

\section{S.T.ONE}
\label{sec:stone}

The \emph{S.T.ONE} loudspeaker define a simplest way to design spherical sound shape. to be continued?

Omnidirectional point of listening as synthesis of sources not space.
Space synthesis through sources synthesis. to be continued?

Photos

Schemes

cad

The first work for S.T.ONE is \emph{Attraverso la lente} (\emph{through the lens)}
an acousmatic work by Giuseppe Silvi (2009) five years before first S.T.ONE loudspeaker
was made. The work came from strong knowledge of principles of tetrahedral sound description,
without the same knowledge on loudspeaker design. When the project was ready, on 2014,
the first stage of development involved a spherical sound diffusion technique of acousmatic
music (\emph{EMUFest 2014}) while a research path concerning the \emph{Sound Shape} of
traditional instruments was taking place.

The \emph{Sound Shape} is the perceived shape of an acoustic object. During  the recording of
\emph{13 Degrees of Darkness} (A. Lucier, for and pre-recorded flute, performed during
EMUFest 2014) the \emph{\textbf{TETRAREC}} technique for \emph{Sound Diffusion} recording
of acoustical instruments was developed.

With \emph{S.T.ONE} and \emph{TETRAREC} a new approach to electroacoustic music is possible
for a better integration between acoustical instruments and electronics.
 
\emph{S.T.ONE} is a product of this research, where live performance with balance between
acoustical and electronics is complete, poly-dimensional and completely new to audience.

With \emph{S.T.ONE} loudspeaker you could control sound reproduction in each direction of
space and integrate each control on a dedicate music composition approach.

%++++++++++++++++++++++++++++++++++++++++++++++++++++++++++++++++++++++++++++++++++++++++++++

\subsection{Mono to Stereo Strategy}
\label{sec:m2st}

In traditinal panning, or amplitude panning, the natural approach to move mono sounds in a
determinate position is related on array of loudspeaker . 

With STONE a strategy for stereo diffusion of mono sources should be bla bla bla mono eq shape etc

%++++++++++++++++++++++++++++++++++++++++++++++++++++++++++++++++++++++++++++++++++++++++++++

\subsection{Stereo from one point of listening}
\label{sec:stereo}

Consider stereo as attitude of listening permits strategies for sound reproduction without 
predeterminate rules on number of loudspeaker. A single STONE loudspeaker should reproduce stereo
sounds from one point.

%++++++++++++++++++++++++++++++++++++++++++++++++++++++++++++++++++++++++++++++++++++++++++++

\subsection{AtoA module}
\label{sec:atoa}

Gerzon's A-Format and STONE have twisted destinies 

%\begin{figure}[t]
%\centering
%\includegraphics[width=0.6\columnwidth]{figure}
%\caption{Figure captions should be placed below the figure, 
%exactly like this.\label{fig:example}}
%\end{figure}
%
%%\begin{figure}[t]
%%\figbox{
%%\subfloat[][]{\includegraphics[width=60mm]{figure}\label{fig:subfigex_a}}\\
%%\subfloat[][]{\includegraphics[width=80mm]{figure}\label{fig:subfigex_b}}
%%}
%%\caption{Here's an example using the subfig package.\label{fig:subfigex} }
%%\end{figure}

%++++++++++++++++++++++++++++++++++++++++++++++++++++++++++++++
%++++++++++++++++++++++++++++++++++++++++++++++++TETRAREC+++++
%++++++++++++++++++++++++++++++++++++++++++++++++++++++++++++
\section{TETRAREC}
\label{sec:tetrarec}



%++++++++++++++++++++++++++++++++++++++++++++++++++++++++++++++
%++++++++++++++++++++++++++++++++++++++++++++++++TETRAREC+++++
%++++++++++++++++++++++++++++++++++++++++++++++++++++++++++++
\section{SOUND SHAPE}
\label{sec:sshape}

On sound shape concept.

\subsection{Chromatic Shape}
\label{sec:cshape}

Plot della scala cromatica di un sax e di un flauto

\subsection{Density 21.5 Shape}
\label{sec:dshape}

Plot di density

\subsection{A Berio Shape}
\label{sec:bshape}

plot della sequenza IXb

%\begin{figure}[t]
%\centering
%\includegraphics[width=0.6\columnwidth]{figure}
%\caption{Figure captions should be placed below the figure, 
%exactly like this.\label{fig:example}}
%\end{figure}
%
%%\begin{figure}[t]
%%\figbox{
%%\subfloat[][]{\includegraphics[width=60mm]{figure}\label{fig:subfigex_a}}\\
%%\subfloat[][]{\includegraphics[width=80mm]{figure}\label{fig:subfigex_b}}
%%}
%%\caption{Here's an example using the subfig package.\label{fig:subfigex} }
%%\end{figure}

%++++++++++++++++++++++++++++++++++++++++++++++++++++++++++++++
%++++++++++++++++++++++++++++++++++++++++++++++++++++PS01+++++
%++++++++++++++++++++++++++++++++++++++++++++++++++++++++++++
\section{\emph{PS: Song \#01}. Case Study.}
\label{sec:ps01}

pasquale citera compila questa parte musicale

il materiale audio sia elettronico (bformat) che tetraedrico pu� essere analizzato da marco.

%\section{Introduction}\label{sec:introduction}
%This template includes all the information about formatting manuscripts for 
%the SMC2017 Conference.
%Please use \LaTeX{} templates when 
%preparing your submission.
%Please follow these guidelines to give the final proceedings a professional look.
%If you have any questions, please contact the SMC2017 organizers.
%This template can be downloaded from:
%\url{http://smc2017.aalto.fi}.
%
%\section{Page size and format}\label{sec:page_size}
%Your paper must not exceed {\bf 8 pages},
%no matter if you are presenting orally or posterly.
%We \underline{strongly encourage}
%a paper length of {\bf 6 pages}.
%We will format the proceedings as
% \underline{portrait A4-size paper} \underline{(21.0cm x 29.7cm)}.
%All material on each page should fit within a rectangle of 17.2cm x 25.2cm,
%centred on the page, beginning 2.0cm from the top of the page and ending 
%with 2.5cm from the bottom.
%The left and right margins should be 1.9cm.
%The text should be in two 8.2cm columns with a 0.8cm gutter.
%All text must be in a two-column format, and justified.
%If you prepare your document by cutting and pasting into this one,
%then you should not have to worry, 
%unless there is something strange with your \LaTeX{} interpreter.
%So double check.
%If you have any questions, please contact the SMC2016 organizers.
%
%\section{Typeset Text}\label{sec:typeset_text}
%
%\subsection{Normal or Body Text}\label{subsec:body}
%Please use a 10pt (point) Times font. 
%Use sans-serif or non-proportional fonts
%only for special purposes, 
%such as distinguishing source code.
%The first paragraph in each section should not be indented, 
%but all other paragraphs should be.
%
%\subsection{Title and Authors}
%As you can see above, the title is 16pt Times, bold, upper case, and centred.
%The names of the authors are also centred.
%The lead author's name is to be listed first (left-most), and the co-authors' 
%names after. If the addresses for all authors are the same, include the 
%address only once, centred. If the authors have different addresses, put the 
%addresses, evenly spaced, under each authors' name.
%
%\subsection{First Page Copyright Notice}
%Please leave the copyright notice exactly as it appears in the lower 
%left-hand corner of the first page. It is set in 8pt Times, if you are wondering.
%
%\subsection{Page Numbering, Headers and Footers}
%Do not include headers, footers or page numbers in your submission.
%We add these electronically when we assemble the publications
%into the proceedings.
%
%\section{Headings}
%First level headings are in Times 12pt bold,
%centred with 1 line of space above the section head, and 1/2 space below it.
%For a section header immediately followed by a subsection header, the space 
%should be merged.
%
%\subsection{Second Level Headings}
%Second level headings are in Times 10pt bold, flush left,
%with 1 line of space above the section head, and 1/2 space below it.
%The first letter of each significant word is capitalized.
%
%\subsubsection{Third Level Headings}
%Third level headings are in Times 10pt italic, flush left,
%with 1/2 line of space above the section head, and 1/2 space below it.
%The first letter of significant words is capitalized.
%
%\subsubsection{Level Headings Beyond the Third}
%We strongly discourage any use of
%more than three levels of headings.
%Also, if you have only one subsection in a section,\footnote{Just like this section.}
%then you should reorganize it into one section.
%
%\section{Floats and equations}
%
%\subsection{Equations}
%Equations of importance, 
%or to which you refer later,
%should be placed on separated lines and numbered.
%The number should be on the right side, in parentheses.
%\begin{equation}
%r=\sqrt[13]{3}
%\label{eq:BP}
%\end{equation}
%Refer to equations like so:
%Equation (\ref{eq:BP}) is of particular 
%interest in Hamburg.
%
%\subsection{Figures, Tables and Captions}
%\begin{table}[t]
% \begin{center}
% \begin{tabular}{|l|l|}
%  \hline
%  String value & Numeric value \\
%  \hline
%  Moin! SMC & 2017 \\
%  \hline
% \end{tabular}
%\end{center}
% \caption{Table captions should be placed below the table, exactly like this,
% but using words different from these.}
% \label{tab:example}
%\end{table}
%
%All artwork must be centred, neat, clean and legible.
%And if you include figures in your paper, instead of artwork,
%please make sure they are centred, neat, clean
%and completely legible without super-resolution imaging.
%All lines should be thick and dark enough to be reproducible
%even by a facsimile machine;
%and figures should not be hand-drawn unless your hand is robotically precise.
%Since the proceedings are distributed in electronic form only, 
%we allow colour to be used in figures;
%but please check that your figures are 
%coherent if they are printed in black-and-white.
%For instance, to make your figures zing in several conditions,
%vary line thickness, style, and colour at the same time.
%
%Numbers and captions of figures and tables always appear below the figure/table.
%Leave 1 line space between the figure or table and the caption.
%Figure and tables are numbered consecutively. 
%Captions should be Times 10pt.
%And try to make your captions sufficiently explain your figures and tables.
%Place tables/figures in text as close to the reference as possible, 
%and preferably at the top of the page.
%
%Always refer to tables and figures in the main text, for example:
%see Fig. \ref{fig:example} and \tabref{tab:example}.
%Figures and tables may extend across both columns to a maximum width of 17.2cm.
%
%Vectorial figures are preferred, e.g., eps.
%When using \texttt{Matlab}, 
%export using either (encapsulated) Postscript or PDF format. 
%In order to optimize readability, 
%the font size of text within a figure should be no smaller than
%that of footnotes (8pt font-size). 
%If you use bitmap figures, make sure that 
%the resolution is high enough for print quality. 
%
%\begin{figure}[t]
%\centering
%\includegraphics[width=0.6\columnwidth]{figure}
%\caption{Figure captions should be placed below the figure, 
%exactly like this.\label{fig:example}}
%\end{figure}
%
%%\begin{figure}[t]
%%\figbox{
%%\subfloat[][]{\includegraphics[width=60mm]{figure}\label{fig:subfigex_a}}\\
%%\subfloat[][]{\includegraphics[width=80mm]{figure}\label{fig:subfigex_b}}
%%}
%%\caption{Here's an example using the subfig package.\label{fig:subfigex} }
%%\end{figure}
%
%\subsection{Footnotes}
%You can indicate footnotes with a number in the text,\footnote{This is a footnote.}
%but try to work the content into the main text.
%Use 8pt font-size for footnotes. 
%Place the footnotes at the bottom of the page 
%on which they appear. 
%Precede the footnote with a 0.5pt horizontal rule.
%
%\section{Citations}
%List all bibliographical references at the end of your paper,
%inside a section named ``REFERENCES''.
%Order and number the references in order of appearance. 
%Do not list references that do not appear in the text.
%Reference numbers in the text should appear within square brackets, such as 
%in~\cite{Someone:00} or~\cite{Someone:00,Someone:04,Someone:09}.
%The reference format is the standard IEEE one. 
%We highly recommend you use BibTeX 
%to generate the reference list.

\section{Conclusions}
To finish your full-length paper, end it with a conclusion;
and after careful editing and a final spell-cheek,
submit it through the Conference Web Submission System. 
\underline{Do not} send papers directly by e-mail.
%
\begin{acknowledgments}
You may acknowledge people, projects, 
funding agencies, etc. 
which can be included after the second-level heading
``Acknowledgments'' (with no numbering).
\end{acknowledgments} 

%%%%%%%%%%%%%%%%%%%%%%%%%%%%%%%%%%%%%%%%%%%%%%%%%%%%%%%%%%%%%%%%%%%%%%%%%%%%%
%bibliography here
\bibliography{ambib/gerzon-michael}

\end{document}
